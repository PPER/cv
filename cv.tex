% adopted from
% https://raw.githubusercontent.com/xdanaux/moderncv/master/examples/template.tex
% thanks: Xavier Danaux (xdanaux@gmail.com).
% author: Yixuan Chen (xlk@umich.edu)


\documentclass[11pt,letterpaper,sans]{moderncv}
% moderncv themes
\moderncvstyle{classic}                             % style options are 'casual' (default), 'classic', 'banking', 'oldstyle' and 'fancy'
\moderncvcolor{black}                               % color options 'black', 'blue' (default), 'burgundy', 'green', 'grey', 'orange', 'purple' and 'red'
%\renewcommand{\familydefault}{\sfdefault}         % to set the default font; use '\sfdefault' for the default sans serif font, '\rmdefault' for the default roman one, or any tex font name
%\nopagenumbers{}                                  % uncomment to suppress automatic page numbering for CVs longer than one page

% adjust the page margins
\usepackage[scale=0.75]{geometry}
%\setlength{\hintscolumnwidth}{3cm}                % if you want to change the width of the column with the dates
%\setlength{\makecvheadnamewidth}{10cm}            % for the 'classic' style, if you want to force the width allocated to your name and avoid line breaks. be careful though, the length is normally calculated to avoid any overlap with your personal info; use this at your own typographical risks...

% personal data
\name{Zhiyi}{Pan}
\title{curriculum vitae}                             % optional, remove / comment the line if not wanted
% \address{street and number}{postcode city}{country}% optional, remove / comment the line if not wanted; the "postcode city" and "country" arguments can be omitted or provided empty
\phone[mobile]{+1~(734)~881~5522}                   % optional, remove / comment the line if not wanted; the optional "type" of the phone can be "mobile" (default), "fixed" or "fax"
% \phone[fixed]{+2~(345)~678~901}
% \phone[fax]{+3~(456)~789~012}
\email{zhiyipan@umich.edu}                              % optional, remove / comment the line if not wanted
% \homepage{blog.xlk.me}                         % optional, remove / comment the line if not wanted
\social[github]{PPER}                              % optional, remove / comment the line if not wanted
% \quote{Some quote}                                 % optional, remove / comment the line if not wanted

\renewcommand*{\bibliographyitemlabel}{[\arabic{enumiv}]}
\begin{document}
\makecvtitle

\section{Education}
\cventry{2019-2020}{Bachelor}{Computer Science Engineering}{University of
  Michigan}{\textit{3.84}}{Dual Degree Program 
  \\ Selected coursework: Advanced Compilers; Program Synthesis; Introduction to Operating Systems; Compiler Construction; Programming Languages}
\cventry{2017-2019}{Bachelor}{Electrical and Computer Engineering}{Shanghai
  Jiaotong University}{\textit{3.55}}{}

% \section{Master thesis}
% \cvitem{title}{\emph{Title}}
% \cvitem{supervisors}{Supervisors}
% \cvitem{description}{Short thesis abstract}

\section{Research Interests}
\cvitem{}{Programming languages, compiler and software engineering}
\section{Research}
\cventry{Nov. 2019 - present}{Research Assistant}{prof. Cyrus Omar}{University of
  Michigan}{Future of Programming Lab, Ann Arbor}{Programming Language
  \begin{itemize}
  \item Worked on a paper for a gradual bidirectional typing inference system - type hole inference. 
  \item Developed undo/redo and history recording features for Hazel, a live functional programming environment featuring typed holes.
  \item Designed and implemented a history panel, which was used for EECS 490 teaching work in UM. 
  \end{itemize}}
\cventry{May 2020 - July 2020}{Research Assistant}{prof. Scott Mahlke}{University of
  Michigan}{}{Compiler Optimization
  \begin{itemize}
  \item Learnt and wrote optimization path by LLVM.
  \item Implemented genetic algorithm to find optimized combination of flags for gcc.
  \end{itemize}}

\section{Academic Experience}
\cventry{2020 Fall}{Instructional Aide (IA)}{EECS 490 Programming Languages}{University
  of Michigan}{}{
  TBD!!!!!!!!!!!
  \begin{itemize}
  \item 250-student upper level technical elective course
  \item duties including holding office hours and lab teaching
  \end{itemize}
}
\cventry{August 2020}{Author}{ICFP 2020}{Student Research Competition Track}{}{Type Hole Inference
\begin{itemize}
  \item Won bronze medal in the student research competition.
  \item Submitted an extended abstract, \textit{Type Hole Inference}.
  \item Participated in the poster session and communicated with ICFPers.
  \item Made a presentation on the final talk session.
  \end{itemize}}


\section{Projects}
\cventry{2020 Fall}{Course Project}{}{}{EECS 583 Advanced Compiler}{
	TBD!!!!!
}
\cventry{2020 Fall}{Course Research Project}{}{}{EECS 598 Program Synthesis: Techniques and Applications}{
	TBD!!!!!
}
\cventry{2019 Winter}{Course Project}{Compiler for Decaf}{EECS 483
  Compiler Construction}{}{
  Implemented and optimized a compiler for Decaf (a strongly-typed, object-oriented language with support for inheritance and
encapsulation), written in C++ and compiled Decaf program into MIPS assembly program.
  \begin{itemize}
  \item Implemented a full-stack compiler from lexical analyzing to code generation with clear static, link and run-time error reporting.
  \item Optimized compiler by register allocation improvement, dead code elimination, constant folding, subexpression elimination, constant propagation and forward copy propagation.
  \end{itemize}
}

\cventry{2019 Winter}{Course Project}{Network File System}{EECS 482
  Operating System}{}{
  Implemented a file system featuring encryption, authentication, failure tolerance and concurrency.
}

\cventry{February 2019 - May 2019}{Online Visual Novel}{SJTU Network and Information Organization}{Shanghai}{}{
	Implemented a visual novel featuring multiple story lines and endings, shopping store, archiving, achievement, HP systems.
  \begin{itemize}
  \item Designed and built front-end pages (Html/Css/Js).
  \item Developed the back-end by Django, created database model and deployed it in the cloud server (Ubuntu).
  \item Revised it into a light framework and released on github.
  \end{itemize}
}

\section{Awards}
\cventry{August 2020}{Bronze}{ICFP 2020 Student Research Competition}{}{}{}
\cventry{April 2019}{Champion}{12th Annual VEX U Robot Skills Challenge World Championship}{The Robotics Education \& Competition Foundation}{Louisville}{Over 30,000 people including more than 1,650 teams from 40 nations participated.
\begin{itemize}
  \item Designed and constructed over 5 versions of robots for competitions with other 8 teammates. 
	\item Wrote daily work log and engineering sheets for the team.
  \end{itemize} 
}
\cventry{April 2019}{Second Place}{12th Annual VEX U Robot World Championship}{The Robotics Education \& Competition Foundation}{Louisville}{}
\cventry{February 2019}{Champion}{VEX Robotics Asia Open}{}{Ningbo, Zhejiang}{}
\cventry{November 2018}{Robot Awards}{Summer Design Expo Best Technology Award}{UM-SJTU JI}{Shanghai}{Built a remote-controlled glass wall cleaning robot}
\cventry{2017-2018}{Scholarship}{Undergraduate Excellent Scholarship}{Shanghai Jiao Tong University}{Shanghai}{}
\section{Programming Languages}
\cvitem{Functional}{OCaml, ReasonML}
\cvitem{Imperative}{C, C++}
\cvitem{Scripting}{Python, Javascript, Shell}
\cvitem{Others}{LaTeX, HTML/CSS, Matlab, Mathematica}

\section{Languages}
TBD!!!!
%\cvitem{English}{Proficient, TOEFL: 110}
% \cvitem{Chinese}{Native speaker}

% \section{Interests}
% \cvitem{Guitar}{Novice}

\end{document}

%% end of file `template.tex'.